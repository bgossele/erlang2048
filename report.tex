\documentclass{article}

\begin{document}

\title{CPL: Erlang assignment}
\author{Brecht Gossel\'e - r0259849}
\maketitle

\section{Task 2: tile failure protocol}

In order to handle failing tiles without the gameplay being discernably disturbed, the tiles send \texttt{\{killed, Id, Value, Merged\}} in the reason clause when they exit. The tiles are linked with the manager process when they're spawned and the manager listens for the exit messages in its main receive-loop. Upon receiving a \texttt{\{'EXIT',\_,\{killed, Id, Value, Merged\}\}} message, the manager simply respawns the tile with the \texttt{Value} and \texttt{Merged} parameters it received. \\This approach seemed to work fine, except for very rare cases when somehow the tile was still down when a message was sent to it. Therefore, I also implemented a function \texttt{sendToTile\textbackslash2} (in \texttt{\textbf{glob2.erl}}), which tries sending a message to a tile, catches the error when the transmission fails, and tries again until it succeeds. All functions that send messages to tiles use this function. This approach makes the game work smoothly, even when the blaster is configured to kill tiles every 2 ms with a chance of 50\%.

\section{Task 3: tile concurrency protocol}

To make the communication between manager and tiles fully concurrent, the tiles send a \texttt{tileReady} message when they finish their computations after an command to move. It does this just before propagating the command to the next tile. The manager counts the incoming \texttt{tileReady} messages until it has received 16 of them.\\
At the same time, the manager also keeps track if it has already received a \texttt{sendData} request from the gui. Only when both conditions are satisfied, the manager launches a collector and sends the collected data to the gui.
\\The number of received \texttt{tileReady} messages as well as whether a \texttt{sendData} request has been received, are passed as a parameter in the \texttt{manageloop} function of the manager.


\end{document}